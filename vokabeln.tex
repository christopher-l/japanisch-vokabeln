\documentclass[11pt]{scrartcl}

\usepackage{xltxtra, setspace}
\usepackage{xeCJK}
\usepackage{hyperref}
\usepackage{xcolor}
\usepackage{multicol}
\usepackage{enumitem}
\usepackage{needspace}
\usepackage[ngerman]{babel}

\setCJKmainfont{Source Han Sans JP Normal}
\setCJKsansfont{Source Han Sans JP Normal}
\setmainfont{Source Sans Pro}

\usepackage[CJK,overlap]{ruby}
\renewcommand{\rubysep}{0ex}
\renewcommand{\rubysize}{0.5}
\setlength{\parindent}{0pt}

% Fix ruby spacing
\let\oldruby\ruby
\def\ruby#1#2{\oldruby{#1}{#2}\futurelet\next\addCJKglue}
\def\addCJKglue{\ifx\next\ruby \CJKglue \fi}

\setcounter{secnumdepth}{0}

\newif\ifenumerate
\makeatletter
\newcommand\vocab{\@ifstar{\enumeratetrue\vocab@}{\enumeratefalse\vocab@}}
\newcommand{\vocab@}[2]{
  \begin{minipage}{\columnwidth}
  \raggedright
  {\Large#1}
  \par
  #2
  \vspace{.8em}
  \end{minipage}}
\makeatother

\newcommand\type[1]{
  {\scriptsize#1}
  \vspace{-.3em}
}

\newcommand\form[1]{
  {\small\color{gray}#1}
}

\newcommand{\translation}[1]{
  \par
  \ifenumerate
  \begin{enumerate}[leftmargin=*,topsep=0em,resume]
    \itemsep0em \item #1
  \end{enumerate}
  \else
  \begin{itemize}[leftmargin=*,topsep=0em,label={},labelsep=0em]
    \itemsep0em \item #1
  \end{itemize}
  \fi}

\newcommand{\example}[2]{
  {\small #1}\\[-.3em]
  {\footnotesize #2}\\}

\newcommand{\comment}[1]{
  {\color{gray}(#1)}
}

\newenvironment{vocabs}[1][]
  {\Needspace{8\baselineskip}
   #1\begin{multicols}{3}}
  {\end{multicols}}

\begin{document}

% \raggedcolumns
\begin{vocabs}[\section{Wendungen}]
\vocab
  {おはよう (ございます)。}
  {\translation{Guten Morgen.}}

\vocab
  {\ruby{今}{こん} \ruby{日}{にち} は。}
  {\translation{\comment{nur Kana} Guten Tag.}}

\vocab
  {\ruby{今}{こん} \ruby{晩}{ばん} は。}
  {\translation{\comment{nur Kana} Guten Abend.}}

\vocab
  {お\ruby{休}{やす}み(なさい)。}
  {\translation{\comment{nur Kana} Gute Nacht.}}

\vocab
  {さようなら。}
  {\translation{Auf Wiedersehen.}}

\vocab
  {またね。}
  {\translation{Tschüs.}}

\vocab
  {\ruby{只}{ただ}\ruby{今}{いま}。}
  {\translation{\emph{beim Zurückkehren}}}

\vocab
  {お\ruby{帰}{かえ}り。}
  {\translation{\emph{wenn jemand zurückkehrt}}}

\vocab
  {\ruby{行}{い}ってきます。}
  {\translation{\emph{beim Gehen}}}

\vocab
  {\ruby{行}{い}ってらっしゃい。}
  {\translation{\emph{wenn jemand geht}}}

\vocab
  {もあった}
  {\usage{V\textsc{-gf} こともあった}
   \translation{früher (regelmäßig) gemacht haben}}

\vocab
  {に\ruby{依}{よ}ると}
  {\translation{\comment{nur Kana} laut \ldots, nach Aussage von \ldots}}

\vocab
  {に\ruby{就}{つ}いて}
  {\translation{\comment{nur Kana} \ldots\ betreffend}}

\vocab
  {お\ruby{陰}{かげ}で}
  {\usage{N のおかげで}
   \translation{\comment{nur Kana} dank \ldots}}

\vocab
  {かも\ruby{知}{し}れない}
  {\translation{\comment{nur Kana} möglicherweise}}

\vocab
  {これから}
  {\translation{von jetzt an}}

\vocab
  {\ruby{乾}{かん}\ruby{杯}{ぱい}}
  {\usage{V\textsc{-te} 乾杯}
   \usage{N に 乾杯}
   \translation{Prost, Auf \ldots!}}

\end{vocabs}

\newpage
\section{Misc}
% \begin{vocabs}
% \vocab
  {など}
  {\type{Suffix}
   \translation{so wie, etc.}}

\vocab*
  {\ruby{為}{ため}}
  {\type{Substantiv}
   \usage{S-\textsc{nh} ため(に)}
   \translation{\comment{nur Kana} damit \ldots}
   \translation{\comment{nur Kana} weil \ldots}}

\vocab
  {\ruby{様}{よう}}
  {\type{Substantiv}
   \usage{S-\textsc{nh} よう}
   \translation{\comment{nur Kana} wie \ldots, als ob \ldots}}

\vocab
  {そうだ}
  {\type{Wendung}
   \usage{S-\textsc{nh} そうだ}
   \translation{es heißt, \ldots}}

\vocab
  {それでも}
  {\translation{trotzdem}}

\vocab
  {だろう}
  {\translation{wahrscheinlich}}

\vocab
  {\ruby{同}{おな}じ}
  {\type{Substantiv}
   \usage{etw.\ と 同じ}
   \translation{so wie \ldots, gleich}}

% \end{vocabs}
\begin{vocabs}[\subsection{Partikel}]
\vocab*
  {と}
  {\translation{und \comment{vollständige Aufzählung}}
   \translation{mit \comment{Person}}
   \translation{oder \comment{Vergleich}}
   \example{お\ruby{茶}{ちゃ}とコーヒーと、
     どちらの\ruby{方}{ほう}が\ruby{好}{す}きですか。}
    {Was mögen Sie lieber, Tee oder Kaffee?}
   \type{Satzpartikel}
   \translation{\comment{mit Grundform} Gedanke, indirekte Rede}
   \example{\ruby{明日}{あした}は\ruby{学}{がっ}\ruby{校}{こう}に\ruby{行}{い}く
     と\ruby{思}{おも}います。}
    {Ich denke, morgen gehe ich zur Schule.}
   }

\end{vocabs}
\begin{vocabs}[\subsection{Zahlen}]
\begin{vocabs}[\section{Zahlen}]

\vocab
  {\ruby{一}{いち} \form{ひと}}
  {\translation{eins}}

\vocab
  {\ruby{二}{に} \form{ふた}}
  {\translation{zwei}}

\vocab
  {\ruby{三}{さん} \form{み}}
  {\translation{drei}}

\vocab
  {\ruby{四}{し} \form{よん, よ}}
  {\translation{vier}}

\vocab
  {\ruby{五}{ご} \form{いつ}}
  {\translation{fünf}}

\vocab
  {\ruby{六}{ろく} \form{む}}
  {\translation{sechs}}

\vocab
  {\ruby{七}{しち} \form{なな}}
  {\translation{sieben}}

\vocab
  {\ruby{八}{はち} \form{や}}
  {\translation{acht}}

\vocab
  {\ruby{九}{きゅう} \form{く, ここの}}
  {\translation{neun}}

\vocab
  {\ruby{十}{じゅう} \form{とお}}
  {\translation{zehn}}

\vocab
  {\ruby{百}{ひゃく}}
  {\translation{hundert}}

\vocab
  {\ruby{千}{せん}}
  {\translation{tausend}}

\vocab
  {\ruby{万}{まん}}
  {\translation{zehntausend}}

\end{vocabs}

\begin{vocabs}[\subsection{Zählworte}]

\vocab
  {つ}
  {\type{Zählwort}
   \translation{\comment{mit \textsc{kun}-Lesung} für neutrale Gegenstände}
   \example{りんご \ruby{三}{みっ}つ}{drei Äpfel}}

\vocab
  {\ruby{個}{こ}}
  {\type{Zählwort}
   \translation{für neutrale Gegenstände}
   \example{\ruby{1}{いっ}\ruby{個}{こ}}{ein(e)}
   \example{\ruby{3}{さん}\ruby{個}{こ} の \ruby{苺}{いちご}}
    {drei Erdbeeren}
   \example{\ruby{何}{なん}\ruby{個}{こ}}{wie viele}}

\vocab
  {\ruby{人}{にん} \form{り}}
  {\type{Zählwort}
   \translation{für Menschen}
   \example{\ruby{一}{ひと}\ruby{人}{り}}{allein}
   \example{\ruby{二}{ふた}\ruby{人}{り}}{zu zweit}
   \example{\ruby{三}{さん}\ruby{人}{にん}}{zu dritt}}

\vocab
  {\ruby{枚}{まい}}
  {\type{Zählwort}
   \translation{für flache Gegenstände}
   \example{\ruby{葉}{は} \ruby{書}{がき} \ruby{一}{いち} \ruby{枚}{まい}}
    {eine Postkarte}}

\vocab
  {\ruby{本}{ほん} \form{ぼん, ぽん}}
  {\type{Zählwort}
   \translation{für runde, lange Gegenstände}
   \example{ボールペン \ruby{一}{いっ} \ruby{本}{ぽん}}
    {ein Kugelschreiber}
   \example{\ruby{何}{なん} \ruby{本}{ぼん}}{wie viele}}

\vocab
  {\ruby{冊}{さつ}}
  {\type{Zählwort}
   \translation{für Bücher}
   \example{ノート \ruby{一}{いっ} \ruby{冊}{さつ}}{ein Notizbuch}}

\vocab
  {\ruby{匹}{ひき}}
  {\type{Zählwort}
   \translation{für kleine Tiere}}

\vocab
  {\ruby{頭}{とう}}
  {\type{Zählwort}
   \translation{für große Tiere}}

\vocab
  {\ruby{度}{ど}}
  {\type{Zählwort}
   \translation{mal}
   \type{Substantiv}
   \translation{Grad}
   \example{もう\ruby{一}{いち}\ruby{度}{ど}}{noch einmal}}

\end{vocabs}

\end{vocabs}
\begin{vocabs}[\subsection{Zählworte}]
\vocab
  {つ}
  {\type{Zählwort}
   \translation{\comment{mit \textsc{kun}-Lesung} für neutrale Gegenstände}
   \example{りんご \ruby{三}{みっ}つ}{drei Äpfel}}

\vocab
  {\ruby{個}{こ}}
  {\type{Zählwort}
   \translation{für neutrale Gegenstände}
   \example{\ruby{1}{いっ}\ruby{個}{こ}}{ein(e)}
   \example{\ruby{3}{さん}\ruby{個}{こ} の \ruby{苺}{いちご}}
    {drei Erdbeeren}
   \example{\ruby{何}{なん}\ruby{個}{こ}}{wie viele}}

\vocab
  {\ruby{人}{にん} \form{り}}
  {\type{Zählwort}
   \translation{für Menschen}
   \example{\ruby{一}{ひと}\ruby{人}{り}}{allein}
   \example{\ruby{二}{ふた}\ruby{人}{り}}{zu zweit}
   \example{\ruby{三}{さん}\ruby{人}{にん}}{zu dritt}}

\vocab
  {\ruby{枚}{まい}}
  {\type{Zählwort}
   \translation{für flache Gegenstände}
   \example{\ruby{葉}{は} \ruby{書}{がき} \ruby{一}{いち} \ruby{枚}{まい}}
    {eine Postkarte}}

\vocab
  {\ruby{本}{ほん} \form{ぼん, ぽん}}
  {\type{Zählwort}
   \translation{für runde, lange Gegenstände}
   \example{ボールペン \ruby{一}{いっ} \ruby{本}{ぽん}}
    {ein Kugelschreiber}
   \example{\ruby{何}{なん} \ruby{本}{ぼん}}{wie viele}}

\vocab
  {\ruby{冊}{さつ}}
  {\type{Zählwort}
   \translation{für Bücher}
   \example{ノート \ruby{一}{いっ} \ruby{冊}{さつ}}{ein Notizbuch}}

\vocab
  {\ruby{匹}{ひき}}
  {\type{Zählwort}
   \translation{für kleine Tiere}}

\vocab
  {\ruby{頭}{とう}}
  {\type{Zählwort}
   \translation{für große Tiere}}

\vocab*
  {\ruby{度}{ど}}
  {\type{Zählwort}
   \translation{mal}
   \type{Substantiv}
   \translation{Grad}
   \example{もう\ruby{一}{いち}\ruby{度}{ど}}{noch einmal}}

\vocab
  {\ruby{目}{め}}
  {\type{Suffix}
   \translation{\comment{nach Zähleinheit} Ordnungszahl}
   \example{\ruby{一}{ひと}つ \ruby{目}{め} の りんご}{der erste Apfel}}


\end{vocabs}
\begin{vocabs}[\subsection{Frageworte}]
\vocab*
  {\ruby{何}{なに} \form{なん}}
  {\type{Pronomen, \textsc{no}-Adjektiv}
   \translation{was}
   \type{Präfix}
   \translation{\comment{vor Zählwort} wie viele}}

\vocab*
  {\ruby{幾}{いく}つ}
  {\type{Adverb}
   \translation{\comment{nur Kana} wie viele}
   \translation{wie alt}}

\vocab
  {\ruby{幾}{いく}ら}
  {\type{Adverb, Substantiv}
   \translation{\comment{nur Kana} wie viel \comment{wie teuer}, wie viele}}

\vocab
  {\ruby{何時}{いつ}}
  {\type{Adverbial, Pronomen}
   \translation{\comment{nur Kana} wann}}

\vocab
  {\ruby{誰}{だれ}}
  {\type{Pronomen, \textsc{no}-Adjektiv}
   \translation{wer}}

\vocab
  {\ruby{何処}{どこ}}
  {\type{Pronomen, \textsc{no}-Adjektiv}
   \translation{\comment{nur Kana} wo}}

\vocab*
  {\ruby{何}{なん}で}
  {\type{Adverb}
   \translation{warum}
   \translation{wie}}

%%%%%%%%%%%%%%%%%%%%%%%%%%%%%%%%%%%%%%%%%%%%%%%%%%%%%%%%%%%%%%%%%%%%%%%

\vocab
  {か}
  {\type{Suffix}
   \translation{\comment{nach Fragewort} irgend}
   \example{\ruby{何}{なに}か \ruby{食}{た}べませんか。}
    {Sollen wir / möchten Sie etwas essen?}}

\vocab*
  {も}
  {\type{Suffix}
   \translation{\comment{nach Fragewort} alle}
   \translation{\comment{nach Fragewort, verneint} nichts}
   \example{いつも}{immer / nie \comment{verneint}}
   \example{なにも\ruby{知}{し}らない}
    {nichts wissen}}

\vocab
  {でも}
  {\type{Suffix}
   \translation{\comment{nach Fragewort} beliebig}
   \example{なんでも\ruby{聞}{き}いて。}
    {Frag, was auch immer du wissen willst.}
   \example{なんでも\ruby{知}{し}らない}
    {nicht alles wissen}
   \example{なんでも\ruby{知}{し}っている}
    {alles wissen}}

\end{vocabs}

\newpage
\begin{vocabs}[\section{Substantive}]
\vocab
  {\ruby{日}{に} \ruby{本}{ほん}}
  {\translation{\comment{auch にっぽん} Japan}}

\vocab
  {\ruby{日}{ひ}}
  {\translation{Tag, Sonne}}

\vocab
  {\ruby{月}{つき}}
  {\translation{Mond, Monat}}

\vocab
  {\ruby{人}{ひと}}
  {\translation{Mensch}}

\vocab
  {\ruby{人}{じん}}
  {\type{Suffix}
   \translation{Landeseinwohner}
   \example{\ruby{日}{に} \ruby{本}{ほん} \ruby{人}{じん}}{Japaner(in)}}

\vocab
  {\ruby{火}{ひ}}
  {\translation{Feuer, Flamme}}

\vocab
  {\ruby{水}{みず}}
  {\translation{Wasser}}

\vocab
  {\ruby{木}{き}}
  {\translation{Baum, Holz}}

\vocab*
  {\ruby{金}{かね}}
  {\translation{\comment{oft お金} Geld}
   \translation{Metall}}

\vocab
  {\ruby{土}{つち}}
  {\translation{Erde, Boden, Schmutz, Lehm, Schlamm}}

\vocab
  {\ruby{時}{とき}}
  {\translation{Zeit, Stunde}}

\vocab
  {\ruby{分}{ぶん}}
  {\translation{Anteil, Teil, Verhältnis}}

\vocab
  {\ruby{部}{ぶ}\ruby{分}{ぶん}}
  {\translation{Teil}}

\vocab
  {\ruby{部}{へ}\ruby{屋}{や}}
  {\translation{Zimmer}}

\vocab
  {\ruby{料}{りょう} \ruby{理}{り}}
  {\type{Substantiv, \textsc{suru}-Verb}
   \translation{Speise, kochen}}

\vocab
  {\ruby{食}{しょく} \ruby{事}{じ}}
  {\translation{Mahlzeit}}

\vocab
  {\ruby{意}{い}\ruby{味}{み}}
  {\translation{Bedeutung}}

\vocab*
  {\ruby{気}{き}}
  {\translation{Charakter, Wesen}
   \translation{Aufmerksamkeit}
   \example{etw.\ に\ruby{気}{き}をつける}
   {mit etw.\ vorsichtig sein}}

\vocab
  {\ruby{物}{もの}\ruby{事}{ごと}}
  {\translation{Sache}}

\vocab
  {\ruby{大}{おお}\ruby{本}{もと}}
  {\translation{Ursprung, Basis}}

\vocab
  {\ruby{役}{やく}\ruby{所}{しょ}}
  {\translation{Amt, Behörde}}

\vocab
  {\ruby{成}{せい}\ruby{功}{こう}}
  {\type{Substantiv, \textsc{suru}-Verb}
   \translation{Erfolg}
   \example{\comment{V-\textsc{te}}\ruby{成}{せい}\ruby{功}{こう}する}
    {etw.\ erfolgreich tun}}

\vocab
  {\ruby{警}{けい}\ruby{察}{さつ}}
  {\translation{Polizei}}

\vocab
  {\ruby{趣}{しゅ}\ruby{味}{み}}
  {\translation{Hobby}}

\vocab
  {\ruby{野}{や}\ruby{菜}{さい}}
  {\translation{Gemüse}}

\end{vocabs}

\begin{vocabs}[\subsection{Körper}]
\vocab
  {\ruby{体}{からだ}}
  {\translation{Körper}}

\vocab
  {\ruby{口}{くち}}
  {\translation{Mund}}

\vocab
  {\ruby{目}{め}}
  {\translation{Augen}}

\vocab
  {\ruby{耳}{みみ}}
  {\translation{Ohren}}

\vocab
  {\ruby{手}{て}}
  {\translation{Hand}}

\vocab
  {\ruby{頭}{あたま}}
  {\translation{Kopf}}

\vocab
  {\ruby{髪}{かみ}}
  {\translation{Haare}}

\vocab
  {\ruby{額}{ひたい}}
  {\translation{Stirn}}

\vocab
  {\ruby{顔}{かお}}
  {\translation{Gesicht}}

\vocab
  {\ruby{鼻}{はな}}
  {\translation{Nase}}

\vocab*
  {\ruby{首}{くび}}
  {\translation{Hals}
   \translation{Kopf}}

\vocab
  {\ruby{喉}{のど}}
  {\translation{\comment{nur Kana} Kehle}}

\vocab
  {\ruby{肩}{かた}}
  {\translation{Schulter}}

\vocab
  {\ruby{胸}{むね}}
  {\translation{Brust}}

\vocab
  {\ruby{腕}{うで}}
  {\translation{Arm}}

\vocab
  {\ruby{腹}{はら}}
  {\translation{\comment{auch お\ruby{腹}{なか}} Bauch}}

\vocab
  {\ruby{指}{ゆび}}
  {\translation{Finger, Zehen}}

\vocab*
  {\ruby{足}{あし}}
  {\translation{Fuß}
   \translation{\comment{脚} Bein}}

\vocab
  {お\ruby{尻}{しり}}
  {\translation{Hintern}}

\vocab
  {\ruby{肘}{ひじ}}
  {\translation{\comment{nur Kana} Ellenbogen}}

\vocab
  {\ruby{膝}{ひざ}}
  {\translation{Knie}}

\vocab
  {\ruby{背}{せ}\ruby{中}{なか}}
  {\translation{Rücken}}

\end{vocabs}

\begin{vocabs}[\subsection{Tierkreiszeichen}]
\vocab*
  {\ruby{鼠}{ねずみ}}
  {\translation{Maus, Ratte}
   \translation{\comment{\ruby{子}{ね}} 1.~Zeichen im Tierkreis \comment{Ratte}}}

\vocab*
  {\ruby{牛}{うし}}
  {\translation{Rind, Ochse}
   \translation{\comment{丑} 2.~Zeichen im Tierkreis \comment{Kuh}}}

\vocab*
  {\ruby{虎}{とら}}
  {\translation{Tiger}
   \translation{\comment{寅} 3.~Zeichen im Tierkreis \comment{Tiger}}}

\vocab*
  {\ruby{兎}{うさぎ}}
  {\translation{Hase, Kaninchen}
   \translation{\comment{\ruby{卯}{う}} 4.~Zeichen im Tierkreis \comment{Hase}}}

\vocab*
  {\ruby{竜}{たつ}}
  {\translation{Drache}
   \translation{\comment{辰} 5.~Zeichen im Tierkreis \comment{Drache}}}

\vocab*
  {\ruby{蛇}{へび}}
  {\translation{Schlange}
   \translation{\comment{\ruby{巳}{み}} 6.~Zeichen im Tierkreis \comment{Schlange}}}

\vocab*
  {\ruby{馬}{うま}}
  {\translation{Pferd}
   \translation{\comment{午} 7.~Zeichen im Tierkreis \comment{Pferd}}}

\vocab*
  {\ruby{羊}{ひつじ}}
  {\translation{Schaf}
   \translation{\comment{未} 8.~Zeichen im Tierkreis \comment{Schaf}}}

\vocab*
  {\ruby{猿}{さる}}
  {\translation{Affe}
   \translation{\comment{申} 9.~Zeichen im Tierkreis \comment{Affe}}}

\vocab*
  {\ruby{鳥}{とり} }
  {\translation{Vogel, Huhn}
   \translation{\comment{酉} 10.~Zeichen im Tierkreis \comment{Hahn}}}

\vocab*
  {\ruby{犬}{いぬ}}
  {\translation{Hund}
   \translation{\comment{戌} 11.~Zeichen im Tierkreis \comment{Hund}}}

\vocab*
  {\ruby{猪}{いのしし}}
  {\translation{Wildschwein}
   \translation{\comment{亥} 12.~Zeichen im Tierkreis \comment{Wildschwein}}}

\end{vocabs}

\begin{vocabs}[\subsection{Zeitangaben}]
\vocab*
  {\ruby{日}{にち}}
  {\type{Suffix}
   \translation{Tag des Monats}
   \type{Zählwort}
   \translation{Tage}}
\vocab
  {\ruby{日}{にち}\ruby{曜}{よう}\ruby{日}{び}}
  {\translation{Sonntag}}

\vocab
  {\ruby{月}{がつ}}
  {\type{Suffix}
   \translation{Monat}}
\vocab
  {か\ruby{月}{げつ}}
  {\type{Zählwort}
   \translation{Monate}}
\vocab
  {\ruby{月}{げつ}\ruby{曜}{よう}\ruby{日}{び}}
  {\translation{Montag}}

\vocab
  {\ruby{火}{か}\ruby{曜}{よう}\ruby{日}{び}}
  {\translation{Dienstag}}

\vocab
  {\ruby{水}{すい}\ruby{曜}{よう}\ruby{日}{び}}
  {\translation{Mittwoch}}

\vocab
  {\ruby{木}{もく}\ruby{曜}{よう}\ruby{日}{び}}
  {\translation{Donnerstag}}

\vocab
  {\ruby{金}{きん}\ruby{曜}{よう}\ruby{日}{び}}
  {\translation{Freitag}}

\vocab
  {\ruby{土}{ど}\ruby{曜}{よう}\ruby{日}{び}}
  {\translation{Samstag}}

\vocab
  {\ruby{週}{しゅう}}
  {\translation{Woche}}
\vocab
  {\ruby{週}{しゅう} \ruby{間}{かん}}
  {\type{Zählwort}
   \translation{Wochen}}

\vocab*
  {\ruby{年}{とし}}
  {\translation{Jahr}
   \type{Suffix}
   \translation{\comment{どし} Jahr im Tierkreis}}
\vocab*
  {\ruby{年}{ねん}}
  {\type{Suffix}
   \translation{Kalenderjahr}
   \type{Zählwort}
   \translation{Jahre}}

\vocab
  {\ruby{時}{じ}}
  {\type{Suffix}
   \translation{Uhrzeit \comment{Stunde}}}
\vocab*
  {\ruby{時}{じ}\ruby{間}{かん}}
  {\translation{Zeit}
   \type{Zählwort}
   \translation{Stunden}}

\vocab*
  {\ruby{分}{ふん}}
  {\translation{Minute}
   \type{Suffix}
   \translation{Uhrzeit \comment{Minute}}
   \type{Zählwort}
   \translation{Minuten}}

\vocab
  {\ruby{午}{ご}\ruby{前}{ぜん}}
  {\translation{vormittags}}

\vocab
  {\ruby{午}{ご}\ruby{後}{ご}}
  {\translation{nachmittags}}

\vocab*
  {\ruby{前}{まえ}}
  {\translation{\comment{etw.\ の 前 に} bevor}
   \type{Suffix}
   \translation{vor \comment{Dauer}}
   \example{\ruby{5}{ご}\ruby{時}{じ}\ruby{間}{かん}\ruby{前}{まえ}\comment{に}}
    {vor 5 Stunden}}

\vocab
  {\ruby{後}{あと}}
  {\translation{\comment{etw.\ の 後 で} danach}}

\vocab
  {\ruby{後}{ご}}
  {\type{Suffix}
   \translation{nach \comment{Dauer}}
   \example{\ruby{三日}{みっか}\ruby{後}{ご}\comment{に}}{nach 3 Tagen}}

\vocab
  {\ruby{位}{くらい}}
  {\type{Suffix}
   \translation{ungefähr \comment{Dauer}}
   \example{\ruby{一} {いっ} か \ruby{月}{げつ} \ruby{位}{くらい} \ruby{前}{まえ}から}
    {seit etwa einem Monat}}

\vocab
  {\ruby{朝}{あさ}}
  {\translation{Morgen}}

\vocab
  {\ruby{晩}{ばん}}
  {\translation{Abend}}

\vocab
  {\ruby{夜}{よる}}
  {\translation{Abend, Nacht}}

\vocab
  {\ruby{誕}{たん} \ruby{生}{じょう} \ruby{日}{び}}
  {\translation{Geburtstag}}

\vocab
  {\ruby{今日}{きょう}}
  {\translation{heute}}

\vocab
  {\ruby{昨日}{きのう}}
  {\translation{gestern}}

\vocab
  {\ruby{明日}{あした}}
  {\translation{morgen}}

\end{vocabs}

\begin{vocabs}[\subsection{Ortsangaben}]
\vocab
  {\ruby{上}{うえ}}
  {\type{Substantiv (als Suffix), \textsc{no}-Adjektiv}
   \translation{über}
   \example{テーブル の \ruby{上}{うえ} に \ruby{新}{しん}
     \ruby{聞}{ぶん} が あります。}
    {Auf dem Tisch liegt eine Zeitung.}}

\vocab
  {\ruby{下}{した}}
  {\type{Substantiv}
   \translation{unter}}

\vocab
  {\ruby{右}{みぎ}}
  {\type{Substantiv}
   \translation{rechts}}

\vocab
  {\ruby{左}{ひだり}}
  {\type{Substantiv, \textsc{no}-Adjektiv}
   \translation{links}}

\vocab
  {\ruby{前}{まえ}}
  {\translation{vor}}

\vocab
  {\ruby{後}{うし}ろ}
  {\type{Substantiv}
   \translation{\comment{auch \ruby{後}{うしろ}} hinter}}

\vocab
  {\ruby{横}{よこ}}
  {\type{Substantiv}
   \translation{neben, Seite}}

\vocab
  {\ruby{隣}{となり}}
  {\type{Substantiv, \textsc{no}-Adjektiv}
   \translation{nebenan, Nachbar}}

\vocab
  {\ruby{側}{そば}}
  {\type{Substantiv}
   \translation{(sehr) nah}}

\vocab
  {\ruby{近}{ちか}く}
  {\type{Substantiv}
   \translation{nah, Nachbarschaft}}

\vocab
  {\ruby{向}{む}かい}
  {\type{Substantiv, \textsc{no}-Adjektiv}
   \translation{gegenüber}}

\vocab
  {\ruby{中}{なか}}
  {\type{Substantiv}
   \translation{innen}}

\vocab
  {\ruby{間}{あいだ}}
  {\type{Substantiv}
   \translation{zwischen}
   \example{カメラ \ruby{屋}{や} と \ruby{銀}{ぎん} \ruby{行}{こう} の
   \ruby{間}{あいだ} に \ruby{本}{ほん} \ruby{屋}{や} が あります。}
   {Zwischen dem Fotogeschäft und der Bank ist ein Buchladen.}}

\vocab
  {\ruby{遠}{とお}く}
  {\type{Substantiv, \textsc{no}-Adjektiv}
   \translation{weit entfernt}}

\vocab
  {\ruby{外}{そと}}
  {\type{Substantiv}
   \translation{außerhalb}}

\end{vocabs}

\newpage
\section{Verben}
\begin{vocabs}
\vocab
  {だ \form{です} \form{で} \form{だった} \form{ではない}}
  {\type{Hilfsverb}
   \translation{sein}}

\vocab*
  {する \form{します} \form{して} \form{しない}}
  {\type{\textsc{suru}, transitiv}
   \translation{machen, tun}
   \translation{anlegen \comment{Utensilien}}
   \example{ピアスをしている}{Piercingschmuck tragen}
  }

\vocab
  {ある \form{ない}}
  {\type{\textsc{godan}}
   \translation{vorhanden sein}}

\vocab*
  {\ruby{居}{い}る}
  {\type{\textsc{ichidan}, intransitiv}
   \translation{\comment{nur Kana} sein}
   \type{Hilfsverb}
   \translation{\comment{V-\textsc{te}} gerade tun \comment{Verlauf}}
   \translation{\comment{V-\textsc{te}} immer wieder tun \comment{Gewohnheit}}
   \translation{\comment{V-\textsc{te}} wurde getan und bleibt so\comment{Zustand}}
  }

\vocab*
  {\ruby{来}{く}る \form{\ruby{来}{き}ます} \form{\ruby{来}{き}て}
  \form{\ruby{来}{こ}ない}}
  {\type{\textsc{kuru}, intransitiv}
   \translation{kommen}
   \type{Hilfsverb}
   \translation{\comment{V-\textsc{te}} (kurz) machen gehen}}

\vocab
  {\ruby{行}{い}く \form{\ruby{行}{い}って}}
  {\type{\textsc{godan}, intransitiv}
   \translation{gehen}}

\vocab
  {\ruby{帰}{かえ}る}
  {\type{\textsc{godan}, intransitiv}
   \translation{zurückkehren}
   \example{\ruby{帰}{かえ}ってくる}{zurückkommen}
  }

\vocab
  {\ruby{食}{た}べる}
  {\type{\textsc{ichidan}, transitiv}
   \translation{essen}}

\vocab
  {\ruby{飲}{の}む}
  {\type{\textsc{godan}, transitiv}
   \translation{trinken}}

\vocab*
  {\ruby{見}{み}る}
  {\type{\textsc{ichidan}, transitiv}
   \translation{sehen}
   \type{Hilfsverb}
   \translation{\comment{V-\textsc{te}} als Probe tun, mal tun}}

\vocab
  {\ruby{見}{み}せる}
  {\type{\textsc{ichidan}, transitiv}
   \translation{zeigen}}

\vocab
  {\ruby{読}{よ}む}
  {\type{\textsc{godan}, transitiv}
   \translation{lesen}}

\vocab
  {\ruby{話}{はな}す}
  {\type{\textsc{godan}, transitiv}
   \translation{sprechen, unterhalten}}

\vocab
  {\ruby{会}{あ}う}
  {\type{\textsc{godan}, intransitiv}
   \translation{treffen}}

\vocab
  {\ruby{教}{おし}える}
  {\type{\textsc{ichidan}, transitiv}
   \translation{unterrichten, \comment{jmd.\ に etw.} sagen}}

\vocab*
  {\ruby{起}{お}きる}
  {\type{\textsc{ichidan}, intransitiv}
   \translation{aufstehen}
   \translation{aufwachen}}

\vocab
  {\ruby{開}{あ}ける}
  {\type{\textsc{ichidan}, transitiv}
   \translation{öffnen}}

\vocab
  {\ruby{閉}{し}める}
  {\type{\textsc{ichidan}, transitiv}
   \translation{schließen}}

\vocab
  {\ruby{上}{あ}げる}
  {\type{\textsc{ichidan}, transitiv}
   \translation{geben \comment{nach oben}}}

\vocab
  {\ruby{呉}{く}れる}
  {\type{\textsc{ichidan}, transitiv}
   \translation{\comment{nur Kana} geben \comment{nach unten}}}

\vocab
  {\ruby{貰}{もら}う}
  {\type{\textsc{godan}, transitiv}
   \translation{\comment{nur Kana} erhalten}}

\vocab*
  {\ruby{使}{つか}う}
  {\type{\textsc{godan}, transitiv}
   \translation{verwenden, benutzen}
   \translation{ausgeben \comment{Geld}}}

\vocab
  {\ruby{点}{つ}ける}
  {\type{\textsc{ichidan}, transitiv}
   \translation{anschalten}}

\vocab
  {\ruby{消}{け}す}
  {\type{\textsc{godan}, transitiv}
   \translation{ausschalten}}

\vocab
  {\ruby{知}{し}る}
  {\type{\textsc{godan}, transitiv}
   \translation{Wissen erwerben \comment{insb.\ \ruby{知}{し}っている, \ruby{知}{し}らない}}}

\vocab*
  {\ruby{分}{わ}かる}
  {\type{\textsc{godan}, intransitiv}
   \translation{verstehen}
   \translation{herausfinden, wissen \comment{insb.\ \ruby{分}{わ}からない}}}

\vocab
  {\ruby{思}{おも}う}
  {\type{\textsc{godan}, transitiv}
   \translation{denken, bedenken}}

\vocab
  {\ruby{思}{おも} い \ruby{出}{だ}す}
  {\type{\textsc{godan}, transitiv}
   \translation{erinnern}}

\vocab*
  {\ruby{考}{かんが}える}
  {\type{\textsc{ichidan}, transitiv}
   \translation{denken, überlegen}
   \translation{als etw.\ ansehen, für etw.\ halten}
   \example{\ruby{難}{むずか}しいと\ruby{考}{かんが}える}
    {für schwierig halten}}

\vocab*
  {\ruby{仕}{し}\ruby{舞}{ま}う}
  {\type{\textsc{godan}, transitiv}
   \translation{\comment{nur Kana} aufräumen, zurückbringen}
   \type{Hilfsverb}
   \translation{\comment{V-\textsc{te}} unerwartet, versehentlich,
    bedauerlicherweise tun}
   \translation{\comment{V-\textsc{te}} fertig bringen}
   \example{チョコレート、もう\ruby{全}{ぜん}\ruby{部}{ぶ}
    \ruby{食}{た}べてしまいました}
    {die ganze Schokolade aufgegessen}}

\vocab*
  {\ruby{置}{お}く}
  {\type{\textsc{godan}, transitiv}
   \translation{hinlegen, ablegen}
   \type{Hilfsverb}
   \translation{\comment{V-\textsc{te}} vorsorglich tun}}

\vocab*
  {\ruby{取}{と}る}
  {\type{\textsc{godan}, transitiv}
   \translation{(weg-) nehmen}
   \translation{\comment{auch 脱る} abnehmen \comment{Kopfbedeckung, Utensilien}}
   \translation{\comment{撮る} aufnehmen}
   \example
     {\ruby{写}{しゃ}\ruby{真}{しん} を \ruby{撮}{と}る}
     {ein Foto machen}
  }

\vocab
  {\ruby{回}{まわ}る}
  {\type{\textsc{godan}, intransitiv}
   \translation{sich drehen}}

\vocab
  {\ruby{止}{と}める}
  {\type{\textsc{ichidan}, transitiv}
   \translation{anhalten}}

\vocab
  {\ruby{着}{つ}く}
  {\type{\textsc{godan}, intransitiv}
   \translation{ankommen}}

\vocab
  {\ruby{動}{うご}く}
  {\type{\textsc{godan}, intransitiv}
   \translation{sich bewegen}}

\vocab
  {\ruby{触}{さわ}る}
  {\type{\textsc{godan}, intransitiv}
   \translation{\comment{etw.\ に} anfassen, berühren}}

\vocab
  {\ruby{捨}{す}てる}
  {\type{\textsc{ichidan}, transitiv}
   \translation{wegwerfen}}

\vocab
  {\ruby{笑}{わら}う}
  {\type{\textsc{godan}, intransitiv}
   \translation{lachen}}

\vocab
  {\ruby{泣}{な}く}
  {\type{\textsc{godan}}
   \translation{weinen}}

\vocab
  {\ruby{噛}{か}む}
  {\type{\textsc{godan}, transitiv}
   \translation{beißen, kauen}}

\vocab
  {\ruby{着}{き}る}
  {\type{\textsc{ichidan}, transitiv}
   \translation{anziehen \comment{Oberkörper}}}

\vocab
  {\ruby{履}{は}く}
  {\type{\textsc{godan}, transitiv}
   \translation{anziehen \comment{Unterleib}}}

\vocab
  {\ruby{被}{かぶ}る}
  {\type{\textsc{godan}, transitiv}
   \translation{aufsetzen \comment{Kopfbedeckung}}}

\vocab
  {\ruby{脱}{ぬ}ぐ}
  {\type{\textsc{godan}, transitiv}
   \translation{ausziehen \comment{Kleidung, Schuhe}}}

\vocab
  {\ruby{外}{はず}す}
  {\type{\textsc{godan}, transitiv}
   \translation{abnehmen \comment{Utensilien}}}

\vocab
  {\ruby{後}{こう}\ruby{悔}{かい}する}
  {\type{\textsc{suru}}
   \translation{bedauern}}

\vocab
  {\ruby{続}{つづ}く}
  {\type{\textsc{godan}, intransitiv}
   \translation{andauern}}

\end{vocabs}

\newpage
\section{Adjektive}
\begin{vocabs}
\vocab
  {\ruby{速}{はや}い}
  {\type{\textsc{i}-Adjektiv}
   \translation{schnell}}

\vocab
  {\ruby{早}{はや}い}
  {\type{\textsc{i}-Adjektiv}
   \translation{früh}}

\vocab
  {\ruby{好}{す}き}
  {\type{\textsc{na}-Adjektiv, Substantiv}
   \translation{lieben, mögen}
   \example{\ruby{私}{わたし}はカラオケが\ruby{好}{す}きです。}
    {Ich mag Karaoke.}}

\vocab
  {\ruby{嫌}{きら}い}
  {\type{\textsc{na}-Adjektiv, Substantiv}
   \translation{nicht mögen, hassen}}

\vocab
  {\ruby{上}{じょう} \ruby{手}{ず}}
  {\type{\textsc{na}-Adjektiv, Substantiv}
   \translation{beherrschen, begabt sein}}

\vocab
  {\ruby{下手}{へた}}
  {\type{\textsc{na}-Adjektiv, Substantiv}
   \translation{nicht beherrschen}}

%%%%%%%%%%%%%%%%%%%%%%%%%%%%%%%%%%%%%%%%%%%%%%%%%%%%%%%%%%%%%%%%%%%%%%%%%%%%%%%

\vocab
  {\ruby{詰}{つま}らない}
  {\type{\textsc{i}-Adjektiv}
   \translation{\comment{nur Kana} langweilig}
   \example{つまらない\ruby{物}{もの}ですが。}{Ist nichts Besonderes,
   (aber\ldots)}}

\vocab
  {\ruby{一緒}{いっしょ}}
  {\type{\textsc{no}-Adjektiv, Substantiv}
   \translation{zusammen}
   \example{\ruby{一緒}{いっしょ}に\ruby{行}{い}きませんか。}
    {Möchten Sie nicht mitkommen?}}

\vocab
  {\ruby{弱}{よわ}い}
  {\type{\textsc{i}-Adjektiv}
   \translation{schwach, verwundbar}}

\vocab
  {\ruby{大}{たい}\ruby{切}{せつ}}
  {\type{\textsc{na}-Adjektiv, Substantiv}
   \translation{wichtig}}

\vocab
  {\ruby{怖}{こわ}い}
  {\type{\textsc{i}-Adjektiv}
   \translation{\comment{etw.\ が} schrecklich, Angst haben vor}}

\vocab
  {\ruby{無}{む}\ruby{理}{り}}
  {\type{\textsc{na}-Adjektiv, Substantiv}
   \translation{unangemessen, unmöglich}}

\vocab
  {\ruby{格}{か}\ruby{好}{っこ}いい}
  {\translation{\comment{nur Kana} gut-aussehend}}

\vocab
  {\ruby{難}{にく}い}
  {\type{Hilfsadjektiv, \textsc{i}-Adjektiv}
   \translation{\comment{nur Kana} \comment{V-\textsc{stamm}} schwierig zu \ldots}}

\end{vocabs}

\end{document}
