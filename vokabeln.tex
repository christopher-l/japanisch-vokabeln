% https://nablux.net/tgp/weblog/2013/03/22/how-typeset-japanese-using-xelatex/
\documentclass[11pt]{scrartcl}

\usepackage{xltxtra, setspace}
\usepackage{xeCJK}
\usepackage{xcolor}
\usepackage{multicol}
\usepackage[ngerman]{babel}

\setCJKmainfont{Source Han Sans JP Normal}
\setCJKsansfont{Source Han Sans JP Normal}
\setmainfont{Source Sans Pro}

\usepackage[CJK,overlap]{ruby}
\renewcommand{\rubysep}{0ex}
\setlength{\parindent}{0pt}

\newcommand{\vocab}[4]{
  \begin{minipage}{\columnwidth}
  {\large#1}\\[-.4em]
  {\scriptsize#2}\\[-2em]
  \raggedright
  \begin{enumerate}\itemsep0em #3\end{enumerate}
  \vspace{0em}
  \end{minipage}
}

\newcommand{\example}[2]{
  \\*[.5em]
  #1\\*[-.3em]
  {\footnotesize #2}
  % \vspace{0em}
}

\newcommand{\comment}[1]{
  {\color{gray}(#1)}
}

\begin{document}

\section*{Verben}

\begin{multicols}{3}
\raggedcolumns

\vocab
  {だ \comment{です} \comment{で}}
  {\translation{TEST}}

\vocab
  {だ \comment{です} \comment{で}}
  {\type{Hilfsverb}
   \translation{sein}}

\vocab*
  {する \comment{します} \comment{して} \comment{しない}}
  {\type{\textsc{suru}, transitiv}
   \translation{machen, tun}
   \translation{anlegen \comment{Utensilien}}
   \example{ピアスをしている}{Piercingschmuck tragen}
  }

\vocab
  {ある \comment{ない}}
  {\type{\textsc{godan}}
   \translation{vorhanden sein}}

\vocab*
  {\ruby{居}{い}る}
  {\type{\textsc{ichidan}, intransitiv}
   \translation{\comment{nur Kana} sein}
   \type{Hilfsverb}
   \translation{\comment{V-\textsc{te}} gerade tun \comment{Verlauf}}
   \translation{\comment{V-\textsc{te}} immer wieder tun \comment{Gewohnheit}}
   \translation{\comment{V-\textsc{te}} wurde getan und bleibt so\comment{Zustand}}
  }

\vocab*
  {\ruby{来}{く}る \comment{\ruby{来}{き}ます} \comment{\ruby{来}{き}て}
  \comment{\ruby{来}{こ}ない}}
  {\type{\textsc{kuru}, intransitiv}
   \translation{kommen}
   \type{Hilfsverb}
   \translation{\comment{V-\textsc{te}} (kurz) machen gehen}}

\vocab
  {\ruby{行}{い}く \comment{\ruby{行}{い}って}}
  {\type{\textsc{godan}, intransitiv}
   \translation{gehen}}

\vocab
  {\ruby{帰}{かえ}る}
  {\type{\textsc{godan}, intransitiv}
   \translation{zurückkehren}
   \example{\ruby{帰}{かえ}ってくる}{zurückkommen}
  }

\vocab
  {\ruby{食}{た}べる}
  {\type{\textsc{ichidan}, transitiv}
   \translation{essen}}

\vocab
  {\ruby{飲}{の}む}
  {\type{\textsc{godan}, transitiv}
   \translation{trinken}}

\vocab*
  {\ruby{見}{み}る}
  {\type{\textsc{ichidan}, transitiv}
   \translation{sehen}
   \type{Hilfsverb}
   \translation{\comment{V-\textsc{te}} als Probe tun, mal tun}}

\vocab
  {\ruby{見}{み}せる}
  {\type{\textsc{ichidan}, transitiv}
   \translation{zeigen}}

\vocab
    {\ruby{話}{はな}す}
    {\type{\textsc{godan}, transitiv}
     \translation{sprechen, unterhalten}}

\vocab
    {\ruby{会}{あ}う}
    {\type{\textsc{godan}, intransitiv}
     \translation{treffen}}

\vocab
    {\ruby{教}{おし}える}
    {\type{\textsc{ichidan}, transitiv}
     \translation{unterrichten, \comment{jmd.\ に etw.} sagen}}

\vocab*
    {\ruby{起}{お}きる}
    {\type{\textsc{ichidan}, intransitiv}
     \translation{aufstehen}
     \translation{aufwachen}}

\vocab
  {\ruby{開}{あ}ける}
  {\type{\textsc{ichidan}, transitiv}
   \translation{öffnen}}

\vocab
  {\ruby{閉}{し}める}
  {\type{\textsc{ichidan}, transitiv}
   \translation{schließen}}

\vocab
  {\ruby{上}{あ}げる}
  {\type{\textsc{ichidan}, transitiv}
   \translation{geben \comment{nach oben}}}

\vocab
  {\ruby{呉}{く}れる}
  {\type{\textsc{ichidan}, transitiv}
   \translation{\comment{nur Kana} geben \comment{nach unten}}}

\vocab
  {\ruby{貰}{もら}う}
  {\type{\textsc{godan}, transitiv}
   \translation{\comment{nur Kana} erhalten}}

\vocab*
  {\ruby{使}{つか}う}
  {\type{\textsc{godan}, transitiv}
   \translation{verwenden, benutzen}
   \translation{ausgeben \comment{Geld}}}

\vocab
  {\ruby{点}{つ}ける}
  {\type{\textsc{ichidan}, transitiv}
   \translation{anschalten}}

\vocab
  {\ruby{消}{け}す}
  {\type{\textsc{godan}, transitiv}
   \translation{ausschalten}}

\vocab*
  {\ruby{仕}{し}\ruby{舞}{ま}う}
  {\type{\textsc{godan}, transitiv}
   \translation{\comment{nur Kana} aufräumen, zurückbringen}
   \type{Hilfsverb}
   \translation{\comment{V-\textsc{te}} unerwartet, versehentlich,
    bedauerlicherweise tun}
   \translation{\comment{V-\textsc{te}} fertig bringen}
   \example{チョコレート、もう\ruby{全}{ぜん}\ruby{部}{ぶ}
    \ruby{食}{た}べてしまいました}
    {die ganze Schokolade aufgegessen}}

\vocab*
  {\ruby{置}{お}く}
  {\type{\textsc{godan}, transitiv}
   \translation{hinlegen, ablegen}
   \type{Hilfsverb}
   \translation{\comment{V-\textsc{te}} vorsorglich tun}}

\vocab*
  {\ruby{取}{と}る}
  {\type{\textsc{godan}, transitiv}
   \translation{(weg-) nehmen}
   \translation{\comment{auch 脱る} abnehmen \comment{Kopfbedeckung, Utensilien}}
   \translation{\comment{撮る} aufnehmen}
   \example
     {\ruby{写}{しゃ}\ruby{真}{しん} を \ruby{撮}{と}る}
     {ein Foto machen}
  }

\vocab
  {\ruby{回}{まわ}る}
  {\type{\textsc{godan}, intransitiv}
   \translation{sich drehen}}

\vocab
  {\ruby{止}{と}める}
  {\type{\textsc{ichidan}, transitiv}
   \translation{anhalten}}

\vocab
  {\ruby{着}{つ}く}
  {\type{\textsc{godan}, intransitiv}
   \translation{ankommen}}

\vocab
  {\ruby{動}{うご}く}
  {\type{\textsc{godan}, intransitiv}
   \translation{sich bewegen}}

\vocab
  {\ruby{触}{さわ}る}
  {\type{\textsc{godan}, intransitiv}
   \translation{anfassen, berühren}}

\vocab
  {\ruby{笑}{わら}う}
  {\type{\textsc{godan}, intransitiv}
   \translation{lachen}}

\vocab
  {\ruby{泣}{な}く}
  {\type{\textsc{godan}}
   \translation{weinen}}

\vocab
  {\ruby{噛}{か}む}
  {\type{\textsc{godan}, transitiv}
   \translation{beißen, kauen}}

\vocab
  {\ruby{着}{き}る}
  {\type{\textsc{ichidan}, transitiv}
   \translation{anziehen \comment{Oberkörper}}}

\vocab
  {\ruby{履}{は}く}
  {\type{\textsc{godan}, transitiv}
   \translation{anziehen \comment{Unterleib}}}

\vocab
  {\ruby{被}{かぶ}る}
  {\type{\textsc{godan}, transitiv}
   \translation{aufsetzen \comment{Kopfbedeckung}}}

\vocab
  {\ruby{脱}{ぬ}ぐ}
  {\type{\textsc{godan}, transitiv}
   \translation{ausziehen \comment{Kleidung, Schuhe}}}

\vocab
  {\ruby{外}{はず}す}
  {\type{\textsc{godan}, transitiv}
   \translation{abnehmen \comment{Utensilien}}}

\vocab
  {\ruby{考}{かんが}える}
  {\type{\textsc{ichidan}, transitiv}
   \translation{denken, überlegen}}

\vocab
  {\ruby{後}{こう}\ruby{悔}{かい}する}
  {\type{\textsc{suru}}
   \translation{bedauern}}



\end{multicols}

\end{document}
