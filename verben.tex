\vocab
  {だ \comment{です} \comment{で}}
  {Hilfsverb}
  {\item sein}

\vocab*
  {する \comment{します} \comment{して} \comment{しない}}
  {\textsc{suru}, transitiv}
  {\item machen, tun
   \item anlegen \comment{Utensilien}
   \example{ピアスをしている}{Piercingschmuck trangen}}

% \vocab
%   {\ruby{来}{く}る \comment{\ruby{来}{き}ます} \comment{\ruby{来}{き}て}
%   \comment{\ruby{来}{こ}ない}}
%   {\textsc{kuru}, intransitiv}
%   {\item kommen}

% \vocab
%   {\ruby{行}{い}く \comment{\ruby{行}{い}って}}
%   {\textsc{godan}, intransitiv}
%   {\item gehen}

% \vocab
%     {\ruby{帰}{かえ}る}
%     {\textsc{godan}, intransitiv}
%     {\item zurückkehren
%      \example
%       {\ruby{帰}{かえ}ってくる}
%       {zurückkommen}
%     }

% \vocab
%   {ある \comment{ない}}
%   {\textsc{godan}}
%   {\item vorhanden sein}

% \vocab
%   {\ruby{居}{い}る}
%   {\textsc{ichidan}, intransitiv}
%   {\item \comment{nur Kana} sein}

% \vocab
%   {\ruby{食}{た}べる}
%   {\textsc{ichidan}, transitiv}
%   {\item essen}

% \vocab
%   {\ruby{飲}{の}む}
%   {\textsc{godan}, transitiv}
%   {\item trinken}

% \vocab
%   {\ruby{見}{み}る}
%   {\textsc{ichidan}, transitiv}
%   {\item sehen}

% \vocab
%   {\ruby{見}{み}せる}
%   {\textsc{ichidan}, transitiv}
%   {\item zeigen}

% \vocab
%     {\ruby{話}{はな}す}
%     {\textsc{godan}, transitiv}
%     {\item sprechen, unterhalten}

% \vocab
%     {\ruby{会}{あ}う}
%     {\textsc{godan}, intransitiv}
%     {\item treffen}

% \vocab
%     {\ruby{教}{おし}える}
%     {\textsc{ichidan}, transitiv}
%     {\item unterrichten, \comment{jmd.\ に etw.} sagen}

% \vocab
%     {\ruby{起}{お}きる}
%     {\textsc{ichidan}, intransitiv}
%     {\item aufstehen
%      \item aufwachen}

% \vocab
%   {\ruby{開}{あ}ける}
%   {\textsc{ichidan}, transitiv}
%   {\item öffnen}

% \vocab
%   {\ruby{閉}{し}める}
%   {\textsc{ichidan}, transitiv}
%   {\item schließen}

% \vocab
%   {\ruby{上}{あ}げる}
%   {\textsc{ichidan}, transitiv}
%   {\item geben \comment{nach oben}}

% \vocab
%   {\ruby{呉}{く}れる}
%   {\textsc{ichidan}, transitiv}
%   {\item \comment{nur Kana} geben \comment{nach unten}}

% \vocab
%   {\ruby{貰}{もら}う}
%   {\textsc{godan}, transitiv}
%   {\item \comment{nur Kana} erhalten}

% \vocab
%   {\ruby{使}{つか}う}
%   {\textsc{godan}, transitiv}
%   {\item verwenden, benutzen
%   \item ausgeben \comment{Geld}}

% \vocab
%     {\ruby{点}{つ}ける}
%     {\textsc{ichidan}, transitiv}
%     {\item anschalten}

% \vocab
%     {\ruby{消}{け}す}
%     {\textsc{godan}, transitiv}
%     {\item ausschalten}

% \vocab
%   {\ruby{取}{と}る}
%   {\textsc{godan}, transitiv}
%   {\item (weg-) nehmen
%    \item \comment{auch 脱る} abnehmen \comment{Kopfbedeckung, Utensilien}
%    \item \comment{撮る} aufnehmen
%    \example
%      {\ruby{写}{しゃ}\ruby{真}{しん} を \ruby{撮}{と}る}
%      {ein Foto machen}
%   }


% \vocab
%   {\ruby{回}{まわ}る}
%   {\textsc{godan}, intransitiv}
%   {\item sich drehen}

% \vocab
%   {\ruby{止}{と}める}
%   {\textsc{ichidan}, transitiv}
%   {\item anhalten}

% \vocab
%   {\ruby{着}{つ}く}
%   {\textsc{godan}, intransitiv}
%   {\item ankommen}

% \vocab
%   {\ruby{動}{うご}く}
%   {\textsc{godan}, intransitiv}
%   {\item sich bewegen}

% \vocab
%   {\ruby{触}{さわ}る}
%   {\textsc{godan}, intransitiv}
%   {\item anfassen, berühren}

% \vocab
%   {\ruby{笑}{わら}う}
%   {\textsc{godan}, intransitiv}
%   {\item lachen}

% \vocab
%   {\ruby{泣}{な}く}
%   {\textsc{godan}}
%   {\item weinen}

% \vocab
%   {\ruby{噛}{か}む}
%   {\textsc{godan}, transitiv}
%   {\item beißen, kauen}

% \vocab
%   {\ruby{着}{き}る}
%   {\textsc{ichidan}, transitiv}
%   {\item anziehen \comment{Oberkörper}}

% \vocab
%   {\ruby{履}{は}く}
%   {\textsc{godan}, transitiv}
%   {\item anziehen \comment{Unterleib}}

% \vocab
%   {\ruby{被}{かぶ}る}
%   {\textsc{godan}, transitiv}
%   {\item aufsetzen \comment{Kopfbedeckung}}

% \vocab
%   {\ruby{脱}{ぬ}ぐ}
%   {\textsc{godan}, transitiv}
%   {\item ausziehen \comment{Kleidung, Schuhe}}

% \vocab
%     {\ruby{外}{はず}す}
%     {\textsc{godan}, transitiv}
%     {\item abnehmen \comment{Utensilien}}

% \vocab
%   {\ruby{考}{かんが}える}
%   {\textsc{ichidan}, transitiv}
%   {\item denken, überlegen}

% \vocab
%   {\ruby{後}{こう}\ruby{悔}{かい}する}
%   {\textsc{suru}}
%   {\item bedauern}

