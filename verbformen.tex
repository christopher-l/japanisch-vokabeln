\vocab
  {ながら}
  {\type{Partikel}
   \usage{V-\textsc{masu-stamm} + ながら}
   \translation{während}}

\vocab
  {から}
  {\type{Partikel}
   \usage{V-\textsc{te} + から}
   \translation{nachdem}}

\vocab
  {たい}
  {\type{\textsc{i}-Adjektiv}
   \usage{V-\textsc{masu-stamm} + たい}
   \translation{möchten, wollen}}

\vocab
  {すぎる}
  {\type{\textsc{ichidan}-Verb, intransitiv}
   \usage{V-\textsc{masu-stamm} + すぎる}
   \translation{zu viel}
   }

\vocab
  {そう}
  {\type{\textsc{na}-Adjektiv}
   \usage{V-\textsc{masu-stamm} + すぎる}
   \translation{anscheinend}}

\vocab
  {たり}
  {\type{Partikel}
   \usage{V-\textsc{ta} + り、V-\textsc{ta} + り する}
   \translation{\ldots\ und \ldots\ tun}
   }

\vocab
  {もいい}
  {\type{Phrase}
   \usage{V-\textsc{te} もいい}
   \translation{dürfen}}

\vocab
  {いけない}
  {\type{Phrase}
   \usage{V-\textsc{te} は いけない}
   \usage{V-\textsc{te} は だめ だ}
   \translation{nicht dürfen}}

\vocab*
  {ないで}
  {\usage{V-\textsc{nai} + で}
   \translation{ohne \ldots\ zu tun}
   \translation{bitte nicht \ldots}}

