\vocab
  {\ruby{目}{め}}
  {\type{Suffix}
   \translation{\comment{nach Zähleinheit} Ordnungszahl}
   \example{\ruby{一}{ひと}つ \ruby{目}{め} の りんご}{der erste Apfel}}

\vocab
  {\ruby{以}{い}\ruby{上}{じょう}}
  {\type{Suffix}
   \translation{\comment{nach Zähleinheit} mindestens \ldots}}

\vocab
  {\ruby{以}{い}\ruby{下}{か}}
  {\type{Suffix}
   \translation{\comment{nach Zähleinheit} höchstens \ldots}}

\vocab
  {つ}
  {\type{Zählwort}
   \translation{\comment{mit \textsc{kun}-Lesung} für neutrale Gegenstände}
   \example{りんご \ruby{三}{みっ}つ}{drei Äpfel}}

\vocab
  {\ruby{個}{こ}}
  {\type{Zählwort}
   \translation{für neutrale Gegenstände}
   \example{\ruby{1}{いっ}\ruby{個}{こ}}{ein(e)}
   \example{\ruby{3}{さん}\ruby{個}{こ} の \ruby{苺}{いちご}}
    {drei Erdbeeren}
   \example{\ruby{何}{なん}\ruby{個}{こ}}{wie viele}}

\vocab
  {\ruby{人}{にん} \form{り}}
  {\type{Zählwort}
   \translation{für Menschen}
   \example{\ruby{一}{ひと}\ruby{人}{り}}{allein}
   \example{\ruby{二}{ふた}\ruby{人}{り}}{zu zweit}
   \example{\ruby{三}{さん}\ruby{人}{にん}}{zu dritt}}

\vocab
  {\ruby{枚}{まい}}
  {\type{Zählwort}
   \translation{für flache Gegenstände}
   \example{\ruby{葉}{は} \ruby{書}{がき} \ruby{一}{いち} \ruby{枚}{まい}}
    {eine Postkarte}}

\vocab
  {\ruby{本}{ほん} \form{ぼん, ぽん}}
  {\type{Zählwort}
   \translation{für runde, lange Gegenstände}
   \example{ボールペン \ruby{一}{いっ} \ruby{本}{ぽん}}
    {ein Kugelschreiber}
   \example{\ruby{何}{なん} \ruby{本}{ぼん}}{wie viele}}

\vocab
  {\ruby{冊}{さつ}}
  {\type{Zählwort}
   \translation{für Bücher}
   \example{ノート \ruby{一}{いっ} \ruby{冊}{さつ}}{ein Notizbuch}}

\vocab
  {\ruby{匹}{ひき}}
  {\type{Zählwort}
   \translation{für kleine Tiere}}

\vocab
  {\ruby{頭}{とう}}
  {\type{Zählwort}
   \translation{für große Tiere}}

\vocab*
  {\ruby{度}{ど}}
  {\type{Zählwort}
   \translation{mal}
   \type{Substantiv}
   \translation{Grad}
   \example{もう\ruby{一}{いち}\ruby{度}{ど}}{noch einmal}}

\vocab
  {\ruby{軒}{けん}}
  {\type{Zählwort}
   \translation{für Gebäude}
   \example{136\ruby{軒}{けん}の\ruby{家}{いえ}}{136 Häuser}}

\vocab
  {\ruby{件}{けん}}
  {\type{Zählwort}
   \translation{für Angelegenheiten, Sachen}}

\vocab
  {\ruby{歳}{さい}}
  {\type{Zählwort}
   \translation{für Alter}}
